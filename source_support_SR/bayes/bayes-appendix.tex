%====================================================================
\frame{ \frametitle{Monte Carlo: Illustration (1/3)} \label{app:MCillustration}
  
  \paragraph{Example.} $\pi(\theta) = \Ncal(0, 10)$, $g(\theta) = e^\theta$:
  \begin{itemize}
   \item {\tt theta.sample = rnorm(M, mean=0, sd=sqrt(10))}
   \item {\tt mean(exp(theta.sample))}
%    \item   
%    \begin{tabular}{crrrrc}
%     $M$ & 1e3 & 1e4 & 1e5 & 1e6 & truth \\
%     \hline
%     $\widehat{\Esp}(g(\thetabf))$ & 388.27 & 140.06 & 133.08 & 170.40 & 148.41
%   \end{tabular}
  \end{itemize}
}

%====================================================================
\frame{ \frametitle{Monte Carlo: Illustration (2/3)}

  \paragraph{Properties.} 
  \begin{itemize}
   \item Easy to implement
   $$
    \text{\tt mean(exp(rnorm(M, mean=0, sd=sqrt(10))))}
   $$ \\~
   \item \pause Unbiased: $\Esp\left[\widehat{\Esp}(g(\thetabf))\right] = \Esp(g(\thetabf)$ \\~
   \item \pause Precision proportional to $1 / \sqrt{M}$ \\~
   \item \pause Still, very variant in practice (see next)
  \end{itemize}
 }

%====================================================================
\frame{ \frametitle{Monte Carlo: Illustration (3/3)}

  \begin{tabular}{cc}
    \begin{tabular}{p{.5\textwidth}}
     $
     \theta \sim \textcolor{blue}{\Ncal(0, 10)}, 
     \quad
     g(\theta) = \textcolor{red}{e^\theta}
     $
     
     \bigskip
    \begin{tabular}{lrr}
	 & mean  & sd \\ 
	 \hline 
	 1000  & 194.67  & 338.96 \\ 
	 10000  & 139.63  & 47.24 \\ 
	 1e+05  & 155.65  & 86.93 \\ 
	 1e+06  & 147.76  & 15.68 \\ 
	 truth  & 148.41  & -- 
	 \end{tabular}
    \end{tabular}
    & 
    \hspace{-.1\textwidth}
    \begin{tabular}{p{.5\textwidth}}
	 \includegraphics[width=.4\textwidth, height=.4\textheight]{../figs/EspLogNorm-MC} \\
	 \includegraphics[width=.4\textwidth, height=.4\textheight]{../figs/EspLogNorm-MC-log} 	 
    \end{tabular}
  \end{tabular}
}



